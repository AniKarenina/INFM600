\documentclass[12pt]{article}
\usepackage[letterpaper, left=0.5in, right=0.5in, top=1in, bottom=1in]{geometry}
\usepackage[thinlines]{easytable}
\newcommand\tab[1][0.25cm]{\hspace*{#1}}
\usepackage{titling}
\usepackage{tabulary}

\title{INFM 600 Team GitHub Repository}
%\author{Due Week #}
\date{\vspace{-10ex}}%removes date from maketitle display

\begin{document}
	\maketitle
		\renewcommand{\arraystretch}{2}
		\begin{tabulary}{6.5in}{| p{2.75in} | p{2.5in} | c | c |}
			\hline
			\textbf{Component} & \textbf{Comments} & \textbf{Worth} & \textbf{Given}\\
			\hline
			Is the repository organized according to standards defined in class?  & &  1/15 & \\
			\hline
			\textit{Does the repository} include commented data cleaning documentation? & & 2/15 &\\
			\hline
			...include the fully commented analysis script and its outputs? (may be combined with data cleaning) & &  3/15 & \\
			\hline
			...include any slides or other materials used for the presentation? (may not be applicable) & &  1/15 & \\
			\hline
			...include a brief document summarizing contributorship? & & 1/15 & \\
			\hline
			\textit{Does the written summary} include the target audience and decisions your analysis is intended to support? & & 2/15 & \\
			\hline
			...include a brief description of the source data and processing (cleaning and analysis)? & & 2/15 & \\
			\hline
			...include at least one plot and a suitable interpretation, with both relevant to the audience and decisions? & & 2/15 & \\
			\hline
			...present a persuasive argument for a decision based on the results? & & 1/15 & \\
			\hline
		\end{tabulary}

\begin{flushright}
	\textbf{Total:}\tab[3.3cm]
\end{flushright}


	\textbf{General Comments:}

	
\end{document}