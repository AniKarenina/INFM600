\documentclass[]{article}
\usepackage{lmodern}
\usepackage{amssymb,amsmath}
\usepackage{ifxetex,ifluatex}
\usepackage{fixltx2e} % provides \textsubscript
\ifnum 0\ifxetex 1\fi\ifluatex 1\fi=0 % if pdftex
  \usepackage[T1]{fontenc}
  \usepackage[utf8]{inputenc}
\else % if luatex or xelatex
  \ifxetex
    \usepackage{mathspec}
  \else
    \usepackage{fontspec}
  \fi
  \defaultfontfeatures{Ligatures=TeX,Scale=MatchLowercase}
\fi
% use upquote if available, for straight quotes in verbatim environments
\IfFileExists{upquote.sty}{\usepackage{upquote}}{}
% use microtype if available
\IfFileExists{microtype.sty}{%
\usepackage{microtype}
\UseMicrotypeSet[protrusion]{basicmath} % disable protrusion for tt fonts
}{}
\usepackage[unicode=true]{hyperref}
\hypersetup{
            pdfborder={0 0 0},
            breaklinks=true}
\urlstyle{same}  % don't use monospace font for urls
\IfFileExists{parskip.sty}{%
\usepackage{parskip}
}{% else
\setlength{\parindent}{0pt}
\setlength{\parskip}{6pt plus 2pt minus 1pt}
}
\setlength{\emergencystretch}{3em}  % prevent overfull lines
\providecommand{\tightlist}{%
  \setlength{\itemsep}{0pt}\setlength{\parskip}{0pt}}
\setcounter{secnumdepth}{0}
% Redefines (sub)paragraphs to behave more like sections
\ifx\paragraph\undefined\else
\let\oldparagraph\paragraph
\renewcommand{\paragraph}[1]{\oldparagraph{#1}\mbox{}}
\fi
\ifx\subparagraph\undefined\else
\let\oldsubparagraph\subparagraph
\renewcommand{\subparagraph}[1]{\oldsubparagraph{#1}\mbox{}}
\fi

\date{}

\begin{document}

\textbf{Lesson Plan Week 14 INFM 600 December 7 \& 8}

6:00 -- 6:05 Welcome \& overview (5 min)

6:05 -- 6:45 Data mining: definitions, history -- Fayyad et al. KDD
process: how different from project processes? Application areas,
methods, potential for pairing with AI.

6:45 -- 7:10 Activity: Find R packages, tutorials, reference materials
for each technique type; identify what type of data it requires; put
this into a shared GDoc. 20 minutes for activity, 5 minutes to report
out what kind of data your technique type requires.

7:10 -- 7:20 BREAK -- sign up for presentation slots.

7:20 -- 7:40 Modern data mining -- Chen et al. IoT bkg, other new spaces
where data mining is being applied (ask for examples from Canvas, other
examples). Challenges discussed in article. DM + AI = ML

7:40 -- 8:00 Discussion: policy implications of data mining. What does
this mean in different contexts and industries?

8:00 -- 8:20 Activity: project groups -- review the details on different
techniques assembled in GDoc. Which methods could you apply? Why would
you do it or not do it? 15 minutes in groups, 10 minutes to report-out
by method (for this type, which groups could use it, give an example of
what it could do)

8:20 -- 8:30 Evaluation for Jonathan

8:30 end early

\end{document}
