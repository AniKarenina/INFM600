\documentclass[]{article}
\usepackage{lmodern}
\usepackage{amssymb,amsmath}
\usepackage{ifxetex,ifluatex}
\usepackage{fixltx2e} % provides \textsubscript
\ifnum 0\ifxetex 1\fi\ifluatex 1\fi=0 % if pdftex
  \usepackage[T1]{fontenc}
  \usepackage[utf8]{inputenc}
\else % if luatex or xelatex
  \ifxetex
    \usepackage{mathspec}
  \else
    \usepackage{fontspec}
  \fi
  \defaultfontfeatures{Ligatures=TeX,Scale=MatchLowercase}
\fi
% use upquote if available, for straight quotes in verbatim environments
\IfFileExists{upquote.sty}{\usepackage{upquote}}{}
% use microtype if available
\IfFileExists{microtype.sty}{%
\usepackage{microtype}
\UseMicrotypeSet[protrusion]{basicmath} % disable protrusion for tt fonts
}{}
\usepackage[unicode=true]{hyperref}
\hypersetup{
            pdfborder={0 0 0},
            breaklinks=true}
\urlstyle{same}  % don't use monospace font for urls
\IfFileExists{parskip.sty}{%
\usepackage{parskip}
}{% else
\setlength{\parindent}{0pt}
\setlength{\parskip}{6pt plus 2pt minus 1pt}
}
\setlength{\emergencystretch}{3em}  % prevent overfull lines
\providecommand{\tightlist}{%
  \setlength{\itemsep}{0pt}\setlength{\parskip}{0pt}}
\setcounter{secnumdepth}{0}
% Redefines (sub)paragraphs to behave more like sections
\ifx\paragraph\undefined\else
\let\oldparagraph\paragraph
\renewcommand{\paragraph}[1]{\oldparagraph{#1}\mbox{}}
\fi
\ifx\subparagraph\undefined\else
\let\oldsubparagraph\subparagraph
\renewcommand{\subparagraph}[1]{\oldsubparagraph{#1}\mbox{}}
\fi

\date{}

\begin{document}

\textbf{Lesson Plan Week 9 INFM 600 October 26 \& 27 }

• Topics R reading in, basic descriptives

• Readings Quick (n.d.)

• Assignment PBJ Documentation - Individual

• Activities R tutorial; PBJ demo; schedule team meetings

• Guest Lecture Jonathan Brier

6:00 -- 6:05 Welcome \& overview (5 min)

6:05 -- 6:25 PBJ Demo (20-30 min)

- Have class volunteer their procedures to follow while building the PBJ

- Have a volunteer from the class in parallel to you assemble the PBJ
sandwich, but you take a liberal interpretation of the instructions.
This maen you may end up with peanut butter and jelly sides down on the
prep surface (prepare accordingly). Highlight points of deviation and
potential method for correcting the deviation. Compare the sandwiches at
the end to the process and to the student's.

Repeat for second documentation.

Touch on ambiguity, assumptions, experience/familiarity with similar
processes. Connect between the PBJ steps with ways it relates to
datasets, clean procedures and tools, and skills.

Materials:

Peanut butter jar

Peanut butter individual serve tins

squeeze jelly

jelly in a jar

jelly and peanut butter goober (mixed in same jar)

loaf of bread (unsliced)

loaf of bread (sliced)

bagels

English muffins

4 knife

4 fork

4 spoon

6:25 -- 6:40 Activity -- compile PBJ procedure together in class as a
discussion-- correcting for assumptions missed in previous steps and
examples made. Make points connecitng each step to working with data.

-initial datasets and tools

-data cleaning

-data processing

-final product

(up to 4 groups can sign up for in class meeting times

6:40 -- 7:00 Activity -- Content and live Demo follow along -

reading in files R → Rstuido

Manipulation of Files

Terms

Reading in a file

Adding a library (packages)

Using a library

Data manipulation

removing columns

removing entries

combining data

replacing data

Descriptives and Analysis

Mean

Median

Standard Deviation

ANOVA

Git integration

(cheat sheets added in Canvas and announcing important questions with
answers as they come in) attach(dataframe) mention

7:10 -- 7:35 Activity -- pairs -- manipulate the whalers datasets

Use the Whalers dataset to:

Drop the Remarks column

What is the average age of all the sailors with reported ages?

What is the distribution of height of sailors with reported heights in
inches?

Figure out something you might want to do to clean or analyze and try
your analysis. Internet search and R help is your friend.

(Walking around and helping people getting to know Rstudio and answering
questions for those who are lost or giving challenges for those already
familiar with R)

7:35 -- 7:50 BREAK -- Schedule Team meetings over the break

7:50 -- 8:05 Activity -- pairs -- interpretation of Whalers R file used
to clean up the whalers dataset previously used in the class.

Walk around and listen to the groups discussion looking up the

8:05 -- 8:45 In class work time for group projects (flexible as needed
if previous activities go long or short)

\end{document}
