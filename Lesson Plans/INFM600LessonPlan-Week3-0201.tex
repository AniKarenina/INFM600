\documentclass[]{article}
\usepackage{lmodern}
\usepackage{amssymb,amsmath}
\usepackage{ifxetex,ifluatex}
\usepackage{fixltx2e} % provides \textsubscript
\ifnum 0\ifxetex 1\fi\ifluatex 1\fi=0 % if pdftex
  \usepackage[T1]{fontenc}
  \usepackage[utf8]{inputenc}
\else % if luatex or xelatex
  \ifxetex
    \usepackage{mathspec}
  \else
    \usepackage{fontspec}
  \fi
  \defaultfontfeatures{Ligatures=TeX,Scale=MatchLowercase}
\fi
% use upquote if available, for straight quotes in verbatim environments
\IfFileExists{upquote.sty}{\usepackage{upquote}}{}
% use microtype if available
\IfFileExists{microtype.sty}{%
\usepackage{microtype}
\UseMicrotypeSet[protrusion]{basicmath} % disable protrusion for tt fonts
}{}
\usepackage[unicode=true]{hyperref}
\hypersetup{
            pdfborder={0 0 0},
            breaklinks=true}
\urlstyle{same}  % don't use monospace font for urls
\IfFileExists{parskip.sty}{%
\usepackage{parskip}
}{% else
\setlength{\parindent}{0pt}
\setlength{\parskip}{6pt plus 2pt minus 1pt}
}
\setlength{\emergencystretch}{3em}  % prevent overfull lines
\providecommand{\tightlist}{%
  \setlength{\itemsep}{0pt}\setlength{\parskip}{0pt}}
\setcounter{secnumdepth}{0}
% Redefines (sub)paragraphs to behave more like sections
\ifx\paragraph\undefined\else
\let\oldparagraph\paragraph
\renewcommand{\paragraph}[1]{\oldparagraph{#1}\mbox{}}
\fi
\ifx\subparagraph\undefined\else
\let\oldsubparagraph\subparagraph
\renewcommand{\subparagraph}[1]{\oldsubparagraph{#1}\mbox{}}
\fi

\date{}

\begin{document}

\textbf{Lesson Plan Week 3 INFM 600 September 14 \& 15}

6:00 -- 6:05 Welcome \& overview (5 min)

6:05 -- 6:20 Review Info Seeking assignment exercise

6:20 -- 6:35 How data are made: Evans \& Schmalensee, responses from
Canvas

6:35 -- 6:55 Howison et al paper: 10 issues review \& discuss which ones
to watch out for in projects

6:55 -- 7:05 Davis: what makes a proposition, theory, or question
interesting?

7:05 -- 7:20 Activity: in pairs, review RQs you came up w/ for Info
Seeking assignment. Pick one \& reframe in Davis's terms. Report out.

7:20 -- 7:30 Discuss: which RQs did you think most interesting? What did
RQs reveal about assumptions of audience knowledge?

7:30 -- 7:40 BREAK

7:40 -- 8:40 Activity: Developing research questions

\begin{itemize}
\item
  Give col headers \& row types
\item
  What Qs can we answer?
\item
  Do some basic summaries in Excel as live demo
\item
  Think-pair-share: other causal factors for energy use variance \& data
  to investigate it?
\item
  Geographic/local knowledge as an asset: let's compare!
\item
  How I found the data for India
\item
  Groups of 3-4: give links to India source data; what steps do you have
  to take to make a comparison data set?
\item
  Discuss: potential causal factors for energy use variance in India.
  How does personal knowledge influence our questions/hypotheses? What
  are the contextual differences it reveals, e.g., electrification,
  appliance availability. What data to find \& how, for answers?
\end{itemize}

\begin{quote}
8:40 -- 8:45 Wrap-up: some ppl will be asked to pitch data next week,
come ready to work with your team on prioritizing options. Assts will be
returned in class next week. Don't forget to make discussion posts;
there won't be prompts every single week
\end{quote}

\end{document}
